\documentclass{beamer}
% Copyright 2015 by Do Phan Thuan

% Loại mẫu slice
%\usetheme{AnnArbor}
%\usetheme{Antibes}
\usetheme{Boadilla}
%\usetheme{CambridgeUS}
%\usetheme{Hannover}

% Ký tự tiếng Việt
\usepackage[utf8]{vietnam}
\usepackage[utf8]{inputenc}
% Công thức toán
\usepackage{amsmath,amsthm,amssymb,epsfig}
% Chèn ảnh
\usepackage{graphicx}
% Chèn đường dẫn 
\usepackage{url}

% Vẽ đồ thị
\usepackage{pgfplots}

% Insert code
\usepackage{listings}
\lstset{language=C++,
   %keywords={break,case,catch,continue,else,elseif,end,for,function,
   %   global,if,otherwise,persistent,return,switch,try,while},
   basicstyle=\ttfamily,
   keywordstyle=\color{blue},
   commentstyle=\color{red},
   stringstyle=\color{dkgreen},
   frame=lrtb,
   %frame=5 pt,
   numbers=left,
   numberstyle=\tiny\color{gray},
   stepnumber=1,
   numbersep=10pt,
   backgroundcolor=\color{white},
   tabsize=4,
   showspaces=false,
   showstringspaces=false}
% Tô mầu cho bảng
\usepackage{colortbl}


\usepackage{color}

\definecolor{dkgreen}{rgb}{0,0.6,0}
\definecolor{gray}{rgb}{0.5,0.5,0.5}
\definecolor{mauve}{rgb}{0.58,0,0.82}
  
\definecolor{Xanh}{rgb}{0,0.5,1}
\definecolor{Do}{rgb}{1,0.25,0}
\definecolor{Vang}{rgb}{1,1,0}
\definecolor{Datroi}{rgb}{0,0,1}
% Vẽ hình


% multirow
\usepackage{multirow}

\usepackage{pbox}

% Tô mầu cho bảng
\usepackage{colortbl}
\definecolor{Xanh}{rgb}{0,0.5,1}
\definecolor{Do}{rgb}{1,0.25,0}
\definecolor{Vang}{rgb}{1,1,0}
\definecolor{Datroi}{rgb}{0,0,1}

% Một vài ký hiệu thường dùng
\def\R{{\mathbb R}}
\def\N{{\mathbb N}}
\def\X{{\mathcal X}}
\def\Y{{\mathcal Y}}
\def\F{{\mathcal F}}
\def\P{{\mathcal P}}
\def\E{{\mathbb E}}
\def\I{{\mathbb I}}
\def\sign{{\rm sign}}

% Xác định khoảng dãn trong bảng
%\renewcommand\arraystretch{1.6}

% a few macros
\newcommand{\bi}{\begin{itemize}}
\newcommand{\ei}{\end{itemize}}
\newcommand{\ig}{\includegraphics}
\newcommand{\subt}[1]{{\footnotesize \color{subtitle} {#1}}}

% named colors
\definecolor{offwhite}{RGB}{249,242,215}
\definecolor{foreground}{RGB}{255,255,255}
\definecolor{background}{RGB}{24,24,24}
\definecolor{title}{RGB}{107,174,214}
\definecolor{gray}{RGB}{155,155,155}
\definecolor{subtitle}{RGB}{102,255,204}
\definecolor{hilight}{RGB}{22,155,104}
\definecolor{vhilight}{RGB}{255,111,207}
\definecolor{lolight}{RGB}{155,155,155}
%\definecolor{green}{RGB}{125,250,125}

% Minted
%\usepackage{minted}
%\usemintedstyle{monokai}
%\newminted{cpp}{fontsize=\footnotesize}

% Graph styles
\tikzstyle{vertex}=[circle,fill=black!50,minimum size=15pt,inner sep=0pt, font=\small]
\tikzstyle{selected vertex} = [vertex, fill=red!24]
\tikzstyle{edge} = [draw,thick,-]
\tikzstyle{dedge} = [draw,thick,->]
\tikzstyle{weight} = [font=\scriptsize,pos=0.5]
\tikzstyle{selected edge} = [draw,line width=2pt,-,red!50]
\tikzstyle{ignored edge} = [draw,line width=5pt,-,black!20]

%gets rid of bottom navigation bars
\setbeamertemplate{footline}[frame number]{}

%gets rid of bottom navigation symbols
%\setbeamertemplate{navigation symbols}{}

%gets rid of footer
%will override 'frame number' instruction above
%comment out to revert to previous/default definitions
%\setbeamertemplate{footline}{}

% Tác giả, Tiêu đề, vân vân
\title[]{{\huge \bf Hệ gơi ý}\\
  }

\author[]{
Nguyễn Tuấn Đạt\\% \inst{1} 
Đặng Quang Trung\\
}

\institute[]{
%\inst{1}% 
}

\logo{\includegraphics[scale=0.05]{hust.jpg} \vspace{220pt}}

\begin{document}

\begin{frame}
\titlepage
\end{frame}

\begin{frame}{Nội dung}
\tableofcontents
\end{frame}

\section{Mô tả dữ liệu}
\section{Các phương pháp sử dụng}
\begin{frame}{Collaborative Filtering}
\color{vhilight}{ Ý tưởng :}
\begin{itemize}
\item Bước 1: Xét người dùng cần gợi ý phim x. Ta tìm tập N người dùng có tập đánh giá phim tương đồng với người dùng x. 
\item Bước 2: Ước lượng đánh giá của người dùng x với những phim mà anh ấy chưa xem bằng cách dựa vào tập N của x. Sau đó, ta đưa ra t phim có ước lượng cao nhất để gợi ý xem cho người dùng x.
\end{itemize} 
\end{frame}
\begin{frame}{Tìm kiếm tập người dùng tương đồng}
Tính độ tương đồng giữa người dùng x và người dùng y bằng độ đo cosin:
$$sim(x,y)=cos(overrightarrow{r_x},\overrightarrow{r_y})=\dfrac{\overrightarrow{r_x}\bullet\overrightarrow{r_y}}{||\overrightarrow{r_x}||||\overrightarrow{r_y}||}$$
Với mỗi người dùng ta sẽ chọn ra k người dùng gần với x nhất. 
\end{frame}
\begin{frame}{Ước lượng với những phim chưa đánh giá}
Xét người dùng x và bộ phim i $r_xi$ sẽ được ước lượng bằng công thức: 
$$ r_{xi}=\dfrac{\sum_{j\in N(x)} S_{x,j} *r_{ji}}{\sum_{j\in N(x) S_{x,j}}}$$
với : $S_{xj}$ là độ tương đồng của người dùng x và người dùng j

\end{frame}
\begin{frame}{Latent Factor Model}
{\color{vhilight}{ Ý tưởng :}} 
Sử dụng SVD để giảm số chiểu của dữ liệu.
$$R=P*Q^T$$
với Q(item,factor) và P(user,factor);


Ước lượng được tính bằng công thức $$r_{xi}=q_i*p_x$$ 
Hàm đánh giá $$ min_{P,Q} \sum_{i,j\in R} = (R_{ij}-q_j*p_i)^2$$
Để mô hình đúng hơn với S dữ liệu đã bị mất ta thêm vào hàm đánh giá các tham số để mong muốn có kết quả tốt hơn trong việc ước lượng các đánh giá.
$$ min_{P,Q} \sum_{i,j\in R} = (R_{ij}-q_j*p_i)^2+[\lambda_1 \sum_i ||p_i||^2 +\lambda_1 \sum_j ||q_j||^2]$$

\end{frame}
\begin{frame}
Ta sẽ sử dụng SGD để tối thiểu hàm đánh giá:
Ta thu được: \\
Với mỗi $r_{xi}:$
\begin{itemize}
\item $\varepsilon_{xi} =2(r_{xi}-q_i*p_x)$
\item $q_i=q_i+\mu_1(\varepsilon_{xi}*p_x-\lambda_2q_i$
\item $p_x=p_x+\mu_1(\varepsilon_{xi}*q_i-\lambda_2p_x$

\end{itemize}
với $\mu_{1,2}$ là tốc độ học 
\end{frame}
\section{Đánh giá các phương pháp}
\section{Kết quả}
\begin{frame}{Tài liệu tham khảo}
\section*{Tài liệu tham khảo}

\end{frame}
\end{document}